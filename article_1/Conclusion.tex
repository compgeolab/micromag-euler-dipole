\section{Conclusion}

\noindent{We developed an efficient semi-automated method to determine the direction of magnetization of dipolar sources on a microscale, as well as the recovery of their magnetic moment. Being ideal for a reinterpretation for the application of methods of paleomagnetic studies using thin sections of rock samples. This would be an attempt to improve the quality of results obtained by isolating the responses of more reliable recorders of the Earth's geomagnetic field. }

\noindent{We also present a new, faster and cleaner way to solve the Euler equation in determining the positioning of magnetic anomaly sources using a pre-selection of magnetic anomaly source windows based on the Laplacian of the Gaussian applied to total gradient anomaly maps. In this way, reducing the numerous solutions to just one data window per source. After estimating the structural index (N = 3) by approximating the sources generating the magnetic anomaly to spheres/points, the Euler deconvolution is performed and the central position of each source is determined.}

\noindent{Since for the recovery of direction and magnetization we only need to assume that the sources have their central positions known (so we apply Euler deconvolution) and that their magnetizations are uniform. This last premise aligns with the theory of magnetically stable particles, which are the basis of classical paleomagnetism. Also, there is no need for any kind of prior knowledge other than the observed magnetic anomaly, and the structural index of the sources. Therefore, this method can be quickly replicated in a dataset of thin sections of rocks to obtain the distributions of magnetic directions of each source identified in the sample.}

\noindent{The test using a simply synthetic sample shows the great capability of the method by retrieving not only the precisely center positions \ref{AppendixA.1} }
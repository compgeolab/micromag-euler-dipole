\begin{singlespace}
Paleomagnetism is the main tool used in paleogeographic reconstructions, in which directional information of the past magnetic field is retrieved from ferromagnetic grains (l.s). Magnetic minerals have a wide range of size and composition that influence their magnetic properties and, consequently, their ability to record the directions of paleomagnetic fields. The signals obtained with classical paleomagnetic methodologies are vector averages that include all magnetic grains present in the samples, including the stable and unstable carriers. To improve the quality of the magnetic signal, it would be necessary to individually identify the magnetizations of the carriers of remanent magnetization, in order to isolate only the good registers of the geomagnetic field. This would allow obtaining more reliable paleomagnetic information, as well as possibly being effective for the paleomagnetic study of older rocks (e.g., Archean rocks), meteorites and even bodies with complex geological evolution. Therefore, this project aims to apply magnetic microscopy to identify the remanent magnetization directions of stable magnetic grains and, subsequently, invert these data in an attempt to recover the individual paleomagnetic directions of the magnetic carriers present in the thin-sections. If successful, a new precision methodology for paleomagnetic data acquisition will be implemented.

\noindent{\textbf{Keywords}: Paleomagnetism, magnetic microscopy, Euler deconvolution, inversion of magnetic data.}
\end{singlespace}
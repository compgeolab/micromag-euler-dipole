%\lipsum[1]
Paleomagnetic data are usually obtained from whole cylindric samples, where the signal is obtained from the sum of all magnetic moments within the sample volume. This usually includes stable and unstable remanence carriers. We developed a fast, semi-automated algorithm capable of estimating the positioning and magnetization of each ferromagnetic (l.s) source using magnetic microscopy data. This methodology is an overdetermined linear inverse problem using as approximation of point dipolar sources, for both position and magnetic moment estimations. It does not require any type of additional information about the sample or sources. Our algorithm works in three basic steps: (i) we apply image processing techniques to identify and isolate the data windows for each source; (ii) with that window data, the position of the sources can be estimated using Euler deconvolution; and finally (iii) using this positioning information the algorithm is able to estimate magnetic direction and intensity for each source. The tests with simple synthetic data show the high effectiveness of the methodology for recovering the positioning and magnetic information for dipolar and non-dipolar sources. While, the complex synthetic data tries replicate the complexity of real rock data, which also shows the feasibility and robustness of the algorithm to estimate positions and magnetization. This is further proved with real data application by retrieving the expected bimodal isothermal remanent directions induced in the sample. This methodology has the potential to statistically improve the result of paleomagnetic studies. 
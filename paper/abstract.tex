%\lipsum[1]
Paleomagnetic data are usually obtained from whole cylindric samples, where the signal results from the sum of magnetic moments from hundreds of thousands to millions of magnetic particles within the sample volume. This usually includes stable and unstable remanence carriers. Recently, magnetic microscopy techniques allowed the investigation of individual grains by directly imaging their magnetic field. However, the determination of the magnetic moments of individual grains is hindered by the intrinsic ambiguity in the inversion of potential field data. We developed a fast, semi-automated algorithm capable of estimating the positioning and magnetization of each ferromagnetic (l.s) source using only the magnetic microscopy data. This methodology is an overdetermined linear inverse problem using an approximation of point dipolar sources, for both position and magnetic moment estimations. It does not require any type of additional information about the sample or sources. Our algorithm works in three basic steps: (i) we apply image processing techniques to identify and isolate the data windows for each source; (ii) with that window data, the position of the sources can be estimated using Euler deconvolution; and finally (iii) using this positioning information the algorithm is able to estimate magnetic direction and intensity for each source. Sensitivity tests were run to estimate the stability of our routine to the depth of particles, noise-to-signal ratio, and non-dipolarity of the sources. Tests with simple synthetic data show the high effectiveness of the methodology for recovering the positioning and magnetic information for dipolar and non-dipolar sources. Complex synthetic data were devised to emulate real rock data. Results obtained on these models also show the feasibility and robustness of the algorithm to estimate the position and magnetic moment of a large set of particles. This is further proved with real data application by retrieving the expected bimodal isothermal remanent directions induced in the sample. Given its semi-automatic nature, its low processing cost, and the possibility of simultaneous inversion of the magnetic moment of a great number of magnetic particles, the methodology here proposed is a step forward in enabling paleomagnetic applications of magnetic microscopy.
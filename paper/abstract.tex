Paleomagnetic data is collected from bulk samples, containing a mixture of
stable and unstable magnetic particles. Recently, magnetic microscopy
techniques have allowed the examination of individual magnetic grains. However,
accurately determining the magnetic moments of these grains is difficult and
time-consuming due to the inherent ambiguity of the data and the large number
of grains in each image. Here we introduce a fast and semi-automated algorithm
that estimates the position and magnetization of dipolar sources solely based
on the magnetic microscopy data. The algorithm follows a three-step process: 1)
employ image processing techniques to identify and isolate data windows for
each magnetic source; 2) use Euler Deconvolution to estimate the position of
each source; 3) solve a linear inverse problem to estimate the dipole moment of
each source. To validate the algorithm, we conducted synthetic data tests,
including varying particle concentrations and non-dipolarity. The tests show
that our method is able to accurately recover the position and dipole moment of
particles that are at least \qty{15}{\um} apart for a source-sensor separation
of \qty{5}{\um}. For grain concentrations of 6250 grains/mm³, our method is
able to detect over 60\% of the particles present in the data. We applied the
method to real data of a speleothem sample, where it accurately retrieved the
expected directions induced in the sample. The semi-automated nature of our
algorithm, combined with its low processing cost and ability to determine the
magnetic moments of numerous particles, represents a significant advancement in
facilitating paleomagnetic applications of magnetic microscopy.

% Paleomagnetic data are usually obtained from whole cylindric samples, where the signal results from the sum of magnetic moments from hundreds of thousands to millions of magnetic particles within the sample volume. 
% This usually includes both stable and unstable remanence carriers.
% Recently, magnetic microscopy techniques allowed the investigation of individual grains by directly imaging their magnetic field. 
% However, the determination of the magnetic moments of individual grains is hindered by the intrinsic ambiguity in the inversion of potential field data, as well as by the large number of grains found in any one microscopy image. 
% We present a fast, semi-automated algorithm capable of estimating the position and magnetization of each ferromagnetic (l.s) source using only the magnetic microscopy data. 
% Our algorithm works in three steps: (i) we first apply image processing techniques to identify and isolate data window boundaries for each source; (ii) with these window boundaries, the position of the sources is estimated using Euler deconvolution; and finally (iii) using the  position information, the algorithm is able to estimate the magnetic dipole moment direction and intensity for each source through an overdetermined linear inverse problem using a dipolar approximation. 
% The method does not require any type of additional information about the sample or the sources. 
% Sensitivity tests were run to estimate the stability of our routine to the depth of particles, signal-to-noise ratio, and non-dipolarity of the sources. 
% Tests with simple synthetic data show the high effectiveness of the methodology for recovering the position and magnetic information for both dipolar and non-dipolar sources. 
% More complex synthetic data including over 100 different magnetic particles were devised to emulate real rock data. 
% Results obtained on these data also show the feasibility and robustness of the algorithm to semi-automatically estimate the position and magnetic moment of a large number of particles. 
% This is further confirmed through an application to real data in which we are able to retrieve the expected bimodal isothermal remanent directions that were induced in the sample. 
% Given its semi-automatic nature, its low processing cost, and the possibility of simultaneous inversion of the magnetic moment of a great number of magnetic particles, the methodology here proposed is a step forward in enabling paleomagnetic applications of magnetic microscopy.
Paleomagnetic data is collected from bulk samples, containing a mixture of stable and unstable magnetic particles. Recently, magnetic microscopy techniques have allowed the examination of individual magnetic grains. However, accurately determining the magnetic moments of these grains is difficult and time-consuming due to the inherent ambiguity of potential field data and the large number of grains in each image. Here we introduce a fast and semi-automated algorithm that estimates the position and magnetization of each magnetic source solely based on magnetic microscopy data, without requiring any additional information. The algorithm follows a three-step process. Firstly, it employs image processing techniques to identify and isolate data window boundaries for each magnetic source. Secondly, the algorithm uses Euler deconvolution to estimate their positioning. Lastly, the algorithm solves an overdetermined linear inverse problem, using a dipolar approximation, to determine the direction and intensity of the magnetic dipole moment for each source. To validate the algorithm, we conducted tests analyzing particle depth, intensity, and non-dipolarity. Additionally, synthetic data experiments were performed to demonstrate the effectiveness of the methodology in recovering the position and magnetic information of both dipolar and non-dipolar sources. These experiments included complex samples designed to simulate real rock data. Then, we applied it to real data, where it accurately retrieved the expected directions induced in the sample. The semi-automated nature of our algorithm, combined with its low processing cost and ability to simultaneously analyze the magnetic moments of numerous particles, represents a significant advancement in facilitating paleomagnetic applications of magnetic microscopy.
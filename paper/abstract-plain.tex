%\lipsum[1]
We propose a method to find out the direction and strength of the magnetism of very small particles in rocks and other materials. These particles can store information about what the Earth’s magnetic field was like in the past. But not all particles are good recorders of this information, and some may have different directions or strengths stored. That is why it is important to have a method that can identify and measure each particle separately. The method uses a device called quantum diamond microscope, which measures the magnetic field near the surface of the material. Then, the method uses a program to calculate the position and magnetic direction and strength of each particle, using a simple model called dipole. The method was tested with synthetic data, fake sources that we already knew, and real data, and showed good results in most cases. The method has the potential to be useful to study the magnetism of natural materials with more accuracy and detail.